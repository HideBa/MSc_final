\chapter{Result}
\label{chp:result}

\section{Overview}
\label{result:overview}

This chapter presents the results of comprehensive evaluations conducted to assess the performance and suitability of the proposed FlatCitybuf format against existing CityJSON encoding approaches. The evaluation followed three complementary methodologies to provide a holistic understanding of the format's capabilities.

\subsection{Evaluation Methodology}
\label{result:overview:evaluation_methodology}

The assessment framework employed three distinct methodological approaches:

\subsubsection{File Size Comparison}
\label{result:overview:file_size_comparison}

\subsubsection{Local Benchmark Performance}
\label{result:overview:local_benchmark_performance}

Performance benchmarks were conducted on a laptop environment to evaluate the computational efficiency of the encoding format. These benchmarks measured:
\begin{itemize}
  \item Read operation time for files of varying sizes
  \item Memory consumption during processing operations
  \item Storage efficiency through file size comparisons
\end{itemize}

The benchmark utilised the datasets from \citet{ledoux_2024} and additional datasets from PLATEAU, providing direct comparability with previous studies on CityJSON and CityJSONSeq formats. All operations were conducted multiple times to ensure statistical reliability, with warm-up iterations to eliminate caching effects.

\subsubsection{Web-Based Performance}
\label{result:overview:web_based_performance}

To assess real-world application performance in cloud environments, web-based benchmarks were implemented using load testing frameworks to measure:
\begin{itemize}
  \item HTTP request-response cycle duration
  \item Effective throughput under various concurrent load scenarios
  \item Bandwidth utilisation, particularly for partial data retrieval operations
  \item Client-side rendering performance with progressive data loading
  \item Performance of HTTP Range requests for spatial and attribute queries
\end{itemize}

These measurements provide critical insights into the cloud optimisation benefits of the format, particularly regarding selective data retrieval and progressive rendering capabilities.

\subsubsection{System Architecture Analysis}
\label{result:overview:system_architecture_analysis}

A comparative analysis of system architectures evaluated how the proposed format affects:
\begin{itemize}
  \item Architectural complexity reduction potential
  \item Server-side resource requirements
  \item Client-side processing overhead
  \item Interoperability with existing GIS ecosystems
  \item Scalability characteristics for large datasets
\end{itemize}

This qualitative and quantitative analysis examines how the encoding format influences the overall system design, particularly focusing on cloud-based deployments and web mapping applications.

The following sections present detailed results from each evaluation approach, followed by integrated analyses that synthesise findings across methodologies to provide comprehensive insights into the performance characteristics of the FlatCitybuf format.

\section{File Size Comparison}
\label{result:file_size_comparison}

\subsubsection{Dataset}
\label{result:overview:dataset}

\subsection{Filesize comparison}
\label{result:overview:filesize_comparison}

\begin{table*}
  \centering
  \begin{threeparttable}
    \caption{The datasets used for the benchmark. }
    \label{tab:dataset_comparison}
    \small
    \begin{tabular}
      {@{}lccccccrrrcrrrr@{}}\toprule
      &&  \multicolumn{2}{c}{\textbf{dataset}} && \multicolumn{3}{c}{\textbf{size of file}} && \multicolumn{3}{c}{\textbf{vertices/attributes}}   \\
      \cmidrule{3-4} \cmidrule{6-8} \cmidrule{10-12}
      && CityFeatures & CityObjects &  app.\footnotesize ${}^{\text{(a)}}$ && CityJSONSeq & FlatCityBuf & compr.\footnotesize ${}^{\text{(b)}}$ && total & ave per feature \footnotesize ${}^{\text{(c)}}$ & attributes per feature \footnotesize ${}^{\text{(d)}}$ & semantic attributes \footnotesize ${}^{\text{(e)}}$ \\
      \midrule
      \textbf{3DBAG}          && \qty{1110} bldgs & \qty{1110} bldgs && \qty{6.7}{\mega\byte} & \qty{5.9}{\mega\byte} & 12\%  &&     \num{82509} &    \num{4112} & \num{37} & \num{1} \\

      \bottomrule
    \end{tabular}
    \begin{tablenotes}[flushleft]
      \footnotesize
    \item ${}^{\text{(a)}}$ appearance: `tex' is textures stored; `mat' is material stored
    \item ${}^{\text{(b)}}$ compressi{}on factor is $\frac{size(CityJSON) - size(CityJSONSeq)}{size(CityJSON)}$
    \item ${}^{\text{(c)}}$ number of vertices in the largest feature of the stream
    \item ${}^{\text{(d)}}$ percentage of vertices that are used to represent different city objects
    \end{tablenotes}
  \end{threeparttable}
\end{table*}

\section{Benchmark on Local Environment}
\label{result:benchmark_on_local_environment}

This section details the performance evaluation of the proposed FlatCitybuf format in a controlled local environment, focusing on read operations, memory consumption, and processing efficiency.

\subsection{Test Environment}
\label{result:benchmark_on_local_environment:test_environment}

All benchmarks were conducted on a consistent hardware and software configuration to ensure reliable and reproducible results:

\begin{itemize}
  \item \textbf{Hardware:} Apple MacBook Pro with M1 Max chip, 32GB unified memory
  \item \textbf{Operating System:} macOS Sequoia 15.4
  \item \textbf{Filesystem:} APFS (Apple File System)
  \item \textbf{Disk Configuration:} 1TB SSD with approximately 500GB free space
  \item \textbf{Runtime Environment:} Rust 1.75.0, with optimised release builds (-O3 optimisation)
\end{itemize}

To minimise environmental variables affecting the measurements, all tests were performed with:
\begin{itemize}
  \item Minimal background processes running
  \item No active network connections (except where required for web benchmarks)
  \item Consistent thermal conditions (ambient temperature and system cooling)
  \item Power connected to eliminate battery state influence
\end{itemize}

\subsection{Benchmark Methodology}
\label{result:benchmark_on_local_environment:benchmark_methodology}

The benchmark procedure followed a systematic approach to ensure measurement accuracy:

\begin{itemize}
  \item \textbf{Warm-up Phase:} Prior to measurement, each format underwent a 5-second warm-up period with repeated operations to ensure CPU caches were properly primed and JIT optimisations were applied

  \item \textbf{Measurement Iterations:} Each operation was executed 100 times consecutively, with measurements recorded for each iteration

  \item \textbf{Statistical Processing:} Measurements were processed to obtain mean values, standard deviations, and confidence intervals (95\%)

  \item \textbf{Process Isolation:} Each format test was run in a separate process to prevent cross-contamination of memory or cache state

  \item \textbf{File System Cache Control:} Between format tests, file system caches were cleared to ensure fair comparison of I/O performance
\end{itemize}

\subsection{Measurement Parameters}
\label{result:benchmark_on_local_environment:measurement_parameters}

The benchmark captured several key performance indicators:

\begin{itemize}
  \item \textbf{Read Time:} The duration required to deserialise the file and access the complete CityJSON structure, measured in milliseconds

  \item \textbf{Memory Consumption:} Peak Resident Set Size (RSS) during file processing, indicating the maximum memory footprint of the operation

  \item \textbf{CPU Utilisation:} Percentage of CPU resources consumed during the operation, measured as an average across the process lifetime

  \item \textbf{Storage Efficiency:} File size comparison between formats, measured in megabytes and percentage reduction relative to the original CityJSON format

  \item \textbf{Partial Data Access:} For formats supporting it, the time required to access specific city objects without loading the entire dataset
\end{itemize}

For each dataset, these parameters were measured across all encoding formats (CityJSON, CityJSONSeq, CBOR, BSON, and FlatCitybuf) to enable direct comparison. The following section presents the results of these measurements and analyses their implications for format performance.

\subsection{Read Performance Results}
\label{result:benchmark_on_local_environment:read_performance_results}

\subsection{Benchmark over the web}
\label{result:benchmark_on_local_environment:benchmark_over_the_web}

\subsection{System architecture review with proposed method and existing method}
\label{result:overview:system_architecture_review_with_proposed_method_and_existing_method}

\subsection{Performance evaluation}
\label{result:overview:performance_evaluation}

\subsection{Case study}
\label{result:overview:case_study}
