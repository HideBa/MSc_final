
\chapter*{Abstract}
Standardizing data formats for 3D city models is crucial for semantically storing real-world information as permanent records.
CityJSON is a widely adopted OGC standard format for this purpose, and its text sequence variant, CityJSON Text Sequences, has been developed to process 3D city model data in a streaming manner, enabling efficient handling of large datasets.
However, the shift towards cloud-native environments and the increasing demand for handling massive datasets necessitate more efficient data processing methods both system-wide and on the web.
While optimized data formats such as PMTiles, FlatBuffers, Mapbox Vector Tiles, and Cloud Optimized GeoTIFF have been proposed for vector and raster data, options for 3D city models remain limited.
This research aims to explore optimized data formats for CityJSON tailored for cloud-native processing and evaluate their performance and use cases.
Specifically, the study will implement FlatBuffers for CityJSON, incorporating features like spatial indexing, spatial sorting, indexing with attribute values, and partial fetching via HTTP Range requests.
The methodology includes a comprehensive review of existing performance-optimized formats, adapting these formats to enable efficient web-based reading of 3D city data with selective features, and benchmarking their performance.
Successful implementation of this research will enable end-users to download arbitrary extents of 3D city models efficiently.
For developers, the optimized format will allow for single-file containment of entire areas of interest, simplification of cloud architecture, and accelerated processing by software applications.
Ultimately, this work will improve the scalability and usability of 3D city models in cloud environments, supporting advanced urban planning and smart city initiatives.