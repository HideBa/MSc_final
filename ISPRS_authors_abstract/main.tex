% Version 2022-09-2022
% update – 161114 by Ken Arroyo Ohori: made spacing closer to Word template throughout, put proper quotes everywhere, removed spacing that could cause labels to be wrong, added non-breaking and inter-sentence spacing where applicable, removed explicit newlines
% update – 010819 by Dennis Wittich: made spacing and font size closer to Word template, updated references and refernces style
% update – 042319 by Dennis Wittich: font size of captions set to 'small', first author names are shortened, hyphenation fixed
% update – 010620 by Dennis Wittich: Footnotes alignment set to left
% update - 151220 by Clement Mallet: Template adapted for single blind abstract submissions
% update - 060321 by Christian Heipke: Template refined for single blind abstract submissions
% update - 090921 by Christian Heipke: Template refined for single blind abstract submissions
% update - 200922 by Christian Heipke: general template update
% update - 080124 by Christian Heipke: general template update

\documentclass{isprs} % isprs class modified 23-04-2019 (Dennis Wittich)
\usepackage{subfigure}
\usepackage{setspace}
\usepackage{geometry} % added 27-02-2014 Markus Englich
\usepackage{epstopdf}
\usepackage[labelsep=period]{caption}  % added 14-04-2016 Markus Englich - Recommendation by Sebastian Brocks
\usepackage[british]{babel}
\usepackage[hang]{footmisc}
\def\footnotemargin{1em} % added 08-01-2020 Dennis Wittich


%\usepackage[authoryear]{natbib}
%\def\bibhang{0pt}

\geometry{a4paper, top=25mm, left=20mm, right=20mm, bottom=25mm, headsep=10mm, footskip=12mm} % added 27-02-2014 Markus Englich
%\usepackage{enumitem}

%\usepackage{isprs}
%\usepackage[perpage,para,symbol*]{footmisc}

%\renewcommand*{\thefootnote}{\fnsymbol{footnote}}
\captionsetup{justification=centering,font=normal} % thanks to Niclas Borlin 05-05-2016
\captionsetup[figure]{font=small} % added 23-04-2019 Dennis Wittich
\captionsetup[table]{font=small} % added 23-04-2019 Dennis Wittich

\begin{document}

\title{Guidelines for Authors preparing an Abstract to be submitted to ISPRS Events
}

% KAO: Remove extra spacing
\author{
 Orhan Altan\textsuperscript{1}, Ian Dowman\textsuperscript{2}, Florent Lafarge\textsuperscript{3}, Clément Mallet\textsuperscript{4}, Christian Heipke\textsuperscript{5} }

% KAO: Remove extra newline
\address{
	\textsuperscript{1 }ITU, Civil Engineering Faculty, 80626 Maslak Istanbul, Turkey - (oaltan, tozg, kulur, seker)@itu.edu.tr\\
	\textsuperscript{2 }Dept.\ of Geomatic Engineering, University College London, Gower Street, London, WC1E 6BT UK - idowman@ge.ucl.ac.uk\\
	\textsuperscript{3 }Université Côte d’Azur, INRIA – Sophia-Antipolis, France – florent.lafarge@inria.fr\\
	\textsuperscript{4 }Univ. Gustave Eiffel, IGN-ENSG, LaSTIG – Saint-Mandé, France – clement.mallet@ign.fr\\
	\textsuperscript{5 }Institute of Photogrammetry and GeoInformation, Leibniz Universit\"at Hannover, Germany - heipke@ipi.uni-hannover.de\\
}

% If the corresponding author is NOT the final author, always add a % space before the subsequent comma, i.e.
% first author name\textsuperscript{a,}\thanks{Corresponding author} , % second author name \textsuperscript{b}, etc.
% thanks to Niclas Borlin 05-05-2016
% information on the corresponding author should not be used any longer and has been commented out
% C. Heipke, Jan 03,2024

% the use of the information of commissions and working groups should not be used any longer and has been commented out
% C. Heipke, Sept. 20,2022
%\commission{XX, }{YY} %This field is optional. If filled, XX and YY should be replaced by adequate numbers. See https://www2.isprs.org/commissions/
%\workinggroup{XX/YY} %This field is optional.
%\icwg{}   %This field is optional.


\keywords{Manuscripts, Proceedings, ISPRS Archives, ISPRS Annals, Guidelines for Authors, Style guide.}
\maketitle
%\saythanks % added 28-02-2014 Markus Englich

\section{Manuscript}\label{Manuscxript}

% KAO: Sloppy spacing ensures non-overfull lines. Can be removed if this is not an issue.
\sloppy

\subsection{Introduction}
\label{sec:Introduction}
These guidelines are provided for the preparation of \textbf{abstracts submitted to ISPRS events} (Congress, Geospatial Week, Symposia, smaller events such as workshops). These guidelines are provided for the preparation of abstracts submitted to ISPRS events (Congress, Geospatial Week, Symposia, smaller events such as workshops). Abstracts are reviewed in a single-blind process. If this process leads to acceptance of the abstract, subsequently a camera-ready manuscript must be submitted following the guidelines for full papers (but, of course, incl. author names and affiliation). This camera-ready manuscript will be published in The International Archives of the Photogrammetry, Remote Sensing and Spatial Information Sciences, provided it arrives by the due date and corresponds to the guidelines.
An example of a camera-ready manuscript can be found on the ISPRS web site under www.isprs.org/documents/orangebook/app5.aspx.

\subsection{General Instructions}\label{sec:General Instructions}

The abstract must have the following structure:

%\itemize
\begin{enumerate}
\setlength\itemsep{0em}\setlength\parskip{0em}\setlength\topsep{0em}\setlength\partopsep{0em}\setlength\parsep{0em}
\item{Title of the contribution}
\item{Author(s) and affiliation}
\item{Keywords (max. 6)}
\item{Main body (one page max)}
\item{Illustrations (one page max, optional)}
\item{References (8 max, optional)}
\end{enumerate}

Abstracts must contain the title, the author(s) names incl. affiliations, the keywords and a summary of the contribution (research question, relevance, solution, evaluation).

\subsection{Page Layout and Length}\label{sec:Page Layout, Spacing and Margins}

The abstract must be compiled in one column for the title, author information and keywords, and in two columns for all subsequent text. Left and right justified typing is mandatory. All abstracts are limited to a length of two (2) single-spaced pages (A4 size), including figures, tables and references. The font type Times New Roman with a size of 9 pts. is to be used.

% KAO: Remove newline
% KAO: Removed spacing before label: can cause references to be wrong
\begin{table}[h]
	\centering
		\begin{tabular}{|l|c|c|}\hline
			Setting&\multicolumn{2}{c|}{A4 size page}\\\hline
			  &mm&inches\\
			 Top&25&1.0\\
			 Bottom&25&1.0\\
			 Left&20&0.8\\
			 Right&20&0.8\\
			 Column Width&82&3.2\\
			 Column Spacing&6&0.25\\\hline
		\end{tabular}
	\caption{Margin settings for A4 size page.}
\label{tab:Margin_settings}
\end{table}

\subsection{Style Guide}\label{sec:Preparation in electronic form}

To assist authors in preparing their contributions, style guides are provided in Word and LaTeX on the ISPRS web site, see www.isprs.org/documents/orangebook/app5.aspx. Use of these style guides ensures that the paper is correctly formatted and is therefore strongly suggested.


\section{Title, Author Information, Keywords}\label{sec:TITLE AND ABSTRACT BLOCK}

\subsection{Title and Author Information}\label{sec:Title}

The title must appear centred in bold at the top of the first page with a size of 12 pts. Following the title, htype the full author(s) name(s), affiliation and mailing address (including e-mail), centred under the title. In the case of multi-authorship, indicate which author belongs to which organisation.

% the following subsection was deleted, Sept. 20, 2022
%\subsection{ISPRS Affiliation (optional)}\label{sec:ISPRS Affiliation (optional)}

% KAO: Use proper quotes
%For those authors affiliated with a specific Commission and/or Working Group of
%ISPRS, a separate title may be entered. The title should be centred in bold type
%after one blank line below the author's affiliation, i.e. Commission~\#, Working Group~\#.
%The Commission number shall be Roman and the Working Group number should be the Commission
%Roman number, slash, WG Arabic number (see example above).


\subsection{Keywords}\label{sec:Keywords}

% KAO: Use proper quotes and dash
Leave two lines blank, then type \textbf{``Keywords:''} in bold, followed by a maximum of 6 keywords. Note that ISPRS does not provide a set list of keywords. Therefore, include those keywords which you would use to find a paper with content you are preparing.

\section{Main Body of Text}\label{sec:MAIN BODY OF TEXT}

Type text single-spaced with one blank line between paragraphs and
following headings. Start paragraphs flush with left margin.

\subsection{Content}\label{sec:Content}
An abstract provides preliminary information and investigations about a novel approach. Therefore, there is no specific predefined ratio between presentation of the method and of the results.

\subsection{Headings}\label{sec:Headings}

% KAO: Remove explicit newlines in this section
Major headings are to be numbered, centred in bold, preceded and followed by a blank line.\\
We do not recommend using subheadings and subsubheadings with a single page of text.\\

\subsection{Footnotes}\label{sec:Footnotes}

Mark footnotes in the text with a number (1); use consecutive numbers for following footnotes. Place footnotes at the bottom of the page, separated from the text above it by a horizontal line.


\subsection{Figures and Tables}\label{sec:Illustrations and Tables}

\subsubsection{Placement:}\label{sec:Placement}

Figures and tables must be placed in the second page of the abstract. While figures and tables are usually aligned horizontally on the page, large figures and tables can be rotated by 90 degrees. If so, make sure that the top is always on the left-hand side of the page.

\subsubsection{Captions:}\label{sec:Captions}

All captions must be centred directly beneath the illustration. Use single spacing if they
use more than one line. All captions are to be numbered consecutively,
e.g. Figure~1, Figure~2, Figure~3, ..  and Table~1, Table~2, Table~3, ...

% KAO: Remove spacing before label: can cause reference to be wrong
\begin{figure}[ht!]
\begin{center}
		\includegraphics[width=1.0\columnwidth]{figures/test_sites/fig1.eps}
	\caption{Figure placement and numbering.}
\label{fig:figure_placement}
\end{center}
\end{figure}


\subsubsection{Copyright:}\label{sec:Copyright}

% KAO: Inter-sentence spacing
If your article contains any copyrighted illustrations or imagery,
include a statement of copyright such as: \copyright~SPOT Image Copyright 20xx
(fill in year), CNES\@. It is the author's responsibility to obtain any necessary
copyright permission. After publication, your article is distributed under \underline{the Creative
Commons Unported License} and you retain the copyright.

\subsection{Equations, Symbols and Units}\label{sec:Equations, Symbols and Units}

\subsubsection{Equations:}\label{sec:Equations}

Equations must be numbered consecutively throughout the contribution. The equation
number is enclosed in parentheses and placed flush right. Leave one blank lines
before and after equations:


\begin{equation}\label{equ:1}
	x = x_0 -c \frac{X - X_0}{Z - Z_0}; y = y_0 -c \frac{Y - Y_0}{Z - Z_0},
\end{equation}

\begin{tabbing}
where \hspace{0.6cm} \= $c$ = principle distance\\
\> $x,y$ = image coordinates\\
\> $X_0,Y_0, Z_0$ = coordinates of projection centre\\
\> $X, Y, Z$ = object coordinates
\end{tabbing}

\subsubsection{Symbols and Units:}\label{sec:Symbols and Units}
Use the SI (Syst\`{e}me International) Units and Symbols. Unusual characters
or symbols must be explained in a list of nomenclature.

% KAO: Non-breaking space
\subsection{References}\label{sec:References}
References must be cited in the text, thus~\cite{smith1987rep}, and listed in alphabetical order in the reference section. While references are optional, maximum 8 references are permitted for abstract-based submissions. The following arrangements must be used:

% KAO: Use proper quotes and non-breaking space
\subsubsection{References from Journals:}
Journals must be cited like~\cite{smith1987} or~\cite{michalis2008}. Names of journals can be abbreviated according to the ``International List of Periodical Title Word Abbreviations''. In case of doubt, write names in full.

\subsubsection{References from Books:}
Books must be cited like~\cite{foerstner2016}.

\subsubsection{References from other Literature:}
Other literature must be cited like~\cite{smith1987rep} and~\cite{smith2000}.

\subsubsection{References from Websites:}
References from the internet must be cited like~\cite{chan2017} and~\cite{maas2017}. Use ofpersistent identifiers such as the Digital Object Identifier (DOI) rather than URLs is strongly advised. In this case last date of visiting the website can be omitted, as the identifier will not change.

\subsubsection{References from Research Data:}
References from research data must be cited like~\cite{dubayah2013}.

\subsubsection{References from Software Projects:}
References to a software project as a high level container including multiple versions of the software must be cited like~\cite{grass2017}.

\subsubsection{References from Software Versions:}
References to a specific software version must be cited like~\cite{grass2015}.

\subsubsection{References from Software Project Add-ons:}
References to a specific software add-on to a software project must be cited like~\cite{lennert2017}.

\subsubsection{References from Software Repository:}
References from software repositories must be cited like~\cite{gago2016}.

{
	\begin{spacing}{1.17}
		\normalsize
		\bibliography{ISPRSguidelines_authors} % Include your own bibliography (*.bib), style is given in isprs.cls
	\end{spacing}
}

\end{document}
